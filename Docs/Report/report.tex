\documentclass[12pt]{article}

\usepackage[italian]{babel}
\usepackage[a4paper]{geometry}
\usepackage{amsmath, tikz, hyperref, caption, subcaption}
\usepackage[capitalize, italian]{cleveref}

% -- Font settings
\usepackage[math-style=ISO]{unicode-math}
\setmainfont{EB Garamond}%You should have installed the font
\setmathfont{Garamond-Math.otf}[StylisticSet={8,9}]
\setmathfont{Garamond-Math.otf}[range={scr,bfscr}]

\setlength{\parskip}{1.2ex}
\setlength{\parindent}{0em}
\clubpenalty = 100
\widowpenalty = 100

\title{STM32 GameBoy}
\date{\today}
\author{Marco Cutecchia, Edoardo Marangoni \\
\footnotesize \texttt{\{marco.cutecchia, edoardo.marangoni1\}@studenti.unimi.it} \\ }

\begin{document}
\maketitle

\section{Introduzione}
Il GameBoy è una console portatile rilasciata da Nintendo all'inizio degli anni
90 che diede inizio al grandissimo successo dei videogiochi tascabili.
La console vendette più di 118 milioni di unità nel mondo e divenne un fenomeno
culturale ricordato ancora oggi.

Lo scopo del progetto è quello di costruire un'imitazione del GameBoy in grado
di giocare tutti i videogiochi originali, ma con una serie di miglioramenti
grazie ai componenti hardware più moderni. Tra le modifiche al dispositivo
originale che abbiamo implementato troviamo il caricamento dei giochi tramite
microSD, invece che con cartucce, e l'utilizzo di uno schermo a colori e
retroilluminato, al contrario dell'originale schermo a scala di grigi che
diventava impossibile da vedere sotto scarsa luce.

\section{Architettura}
\subsection{Hardware}
%\begin{figure}[h]
%	\begin{subfigure}[t]{0.32\textwidth}
%		\begin{center}
%			\begin{tikzpicture}[x=0.015cm, y=0.015cm, scale=0.5, transform shape]
%					\draw  [color={rgb, 255:red, 0; green, 0; blue, 0 }  ,draw opacity=1 ][fill={rgb, 255:red, 255; green, 255; blue, 255 }  ,fill opacity=1 ] (480,140) -- (640,140) -- (640,960) -- (480,960) -- cycle ;
	\draw  [fill={rgb, 255:red, 255; green, 255; blue, 255 }  ,fill opacity=1 ] (440,150) -- (510,150) -- (510,190) -- (440,190) -- cycle ;
	\draw  [fill={rgb, 255:red, 255; green, 255; blue, 255 }  ,fill opacity=1 ] (440,190) -- (510,190) -- (510,230) -- (440,230) -- cycle ;
	\draw  [fill={rgb, 255:red, 255; green, 255; blue, 255 }  ,fill opacity=1 ] (440,230) -- (510,230) -- (510,270) -- (440,270) -- cycle ;
	\draw  [fill={rgb, 255:red, 255; green, 255; blue, 255 }  ,fill opacity=1 ] (440,270) -- (510,270) -- (510,310) -- (440,310) -- cycle ;
	\draw  [fill={rgb, 255:red, 255; green, 255; blue, 255 }  ,fill opacity=1 ] (440,310) -- (510,310) -- (510,350) -- (440,350) -- cycle ;
	\draw  [fill={rgb, 255:red, 255; green, 255; blue, 255 }  ,fill opacity=1 ] (440,350) -- (510,350) -- (510,390) -- (440,390) -- cycle ;
	\draw  [fill={rgb, 255:red, 255; green, 255; blue, 255 }  ,fill opacity=1 ] (440,390) -- (510,390) -- (510,430) -- (440,430) -- cycle ;
	\draw  [fill={rgb, 255:red, 255; green, 255; blue, 255 }  ,fill opacity=1 ] (440,430) -- (510,430) -- (510,470) -- (440,470) -- cycle ;
	\draw  [fill={rgb, 255:red, 255; green, 255; blue, 255 }  ,fill opacity=1 ] (440,470) -- (510,470) -- (510,510) -- (440,510) -- cycle ;
	\draw  [fill={rgb, 255:red, 255; green, 255; blue, 255 }  ,fill opacity=1 ] (440,510) -- (510,510) -- (510,550) -- (440,550) -- cycle ;
	\draw  [fill={rgb, 255:red, 255; green, 255; blue, 255 }  ,fill opacity=1 ] (440,550) -- (510,550) -- (510,590) -- (440,590) -- cycle ;
	\draw  [fill={rgb, 255:red, 255; green, 255; blue, 255 }  ,fill opacity=1 ] (440,630) -- (510,630) -- (510,670) -- (440,670) -- cycle ;
	\draw  [fill={rgb, 255:red, 255; green, 255; blue, 255 }  ,fill opacity=1 ] (440,670) -- (510,670) -- (510,710) -- (440,710) -- cycle ;
	\draw  [fill={rgb, 255:red, 255; green, 255; blue, 255 }  ,fill opacity=1 ] (440,710) -- (510,710) -- (510,750) -- (440,750) -- cycle ;
	\draw  [fill={rgb, 255:red, 255; green, 255; blue, 255 }  ,fill opacity=1 ] (440,750) -- (510,750) -- (510,790) -- (440,790) -- cycle ;
	\draw  [fill={rgb, 255:red, 255; green, 255; blue, 255 }  ,fill opacity=1 ] (440,790) -- (510,790) -- (510,830) -- (440,830) -- cycle ;
	\draw  [fill={rgb, 255:red, 255; green, 255; blue, 255 }  ,fill opacity=1 ] (440,830) -- (510,830) -- (510,870) -- (440,870) -- cycle ;
	\draw  [fill={rgb, 255:red, 255; green, 255; blue, 255 }  ,fill opacity=1 ] (440,870) -- (510,870) -- (510,910) -- (440,910) -- cycle ;
	\draw  [fill={rgb, 255:red, 255; green, 255; blue, 255 }  ,fill opacity=1 ] (440,910) -- (510,910) -- (510,950) -- (440,950) -- cycle ;
	\draw  [fill={rgb, 255:red, 255; green, 255; blue, 255 }  ,fill opacity=1 ] (610,150) -- (680,150) -- (680,190) -- (610,190) -- cycle ;
	\draw  [fill={rgb, 255:red, 255; green, 255; blue, 255 }  ,fill opacity=1 ] (610,190) -- (680,190) -- (680,230) -- (610,230) -- cycle ;
	\draw  [fill={rgb, 255:red, 255; green, 255; blue, 255 }  ,fill opacity=1 ] (610,230) -- (680,230) -- (680,270) -- (610,270) -- cycle ;
	\draw  [fill={rgb, 255:red, 255; green, 255; blue, 255 }  ,fill opacity=1 ] (610,270) -- (680,270) -- (680,310) -- (610,310) -- cycle ;
	\draw  [fill={rgb, 255:red, 255; green, 255; blue, 255 }  ,fill opacity=1 ] (610,310) -- (680,310) -- (680,350) -- (610,350) -- cycle ;
	\draw  [fill={rgb, 255:red, 255; green, 255; blue, 255 }  ,fill opacity=1 ] (610,350) -- (680,350) -- (680,390) -- (610,390) -- cycle ;
	\draw  [fill={rgb, 255:red, 255; green, 255; blue, 255 }  ,fill opacity=1 ] (610,390) -- (680,390) -- (680,430) -- (610,430) -- cycle ;
	\draw  [fill={rgb, 255:red, 255; green, 255; blue, 255 }  ,fill opacity=1 ] (610,430) -- (680,430) -- (680,470) -- (610,470) -- cycle ;
	\draw  [fill={rgb, 255:red, 255; green, 255; blue, 255 }  ,fill opacity=1 ] (610,470) -- (680,470) -- (680,510) -- (610,510) -- cycle ;
	\draw  [fill={rgb, 255:red, 255; green, 255; blue, 255 }  ,fill opacity=1 ] (610,510) -- (680,510) -- (680,550) -- (610,550) -- cycle ;
	\draw  [fill={rgb, 255:red, 255; green, 255; blue, 255 }  ,fill opacity=1 ] (610,550) -- (680,550) -- (680,590) -- (610,590) -- cycle ;
	\draw  [fill={rgb, 255:red, 255; green, 255; blue, 255 }  ,fill opacity=1 ] (610,590) -- (680,590) -- (680,630) -- (610,630) -- cycle ;
	\draw  [fill={rgb, 255:red, 255; green, 255; blue, 255 }  ,fill opacity=1 ] (610,630) -- (680,630) -- (680,670) -- (610,670) -- cycle ;
	\draw  [fill={rgb, 255:red, 255; green, 255; blue, 255 }  ,fill opacity=1 ] (610,670) -- (680,670) -- (680,710) -- (610,710) -- cycle ;
	\draw  [fill={rgb, 255:red, 255; green, 255; blue, 255 }  ,fill opacity=1 ] (610,710) -- (680,710) -- (680,750) -- (610,750) -- cycle ;
	\draw  [fill={rgb, 255:red, 255; green, 255; blue, 255 }  ,fill opacity=1 ] (610,750) -- (680,750) -- (680,790) -- (610,790) -- cycle ;
	\draw  [fill={rgb, 255:red, 255; green, 255; blue, 255 }  ,fill opacity=1 ] (610,790) -- (680,790) -- (680,830) -- (610,830) -- cycle ;
	\draw  [fill={rgb, 255:red, 255; green, 255; blue, 255 }  ,fill opacity=1 ] (610,830) -- (680,830) -- (680,870) -- (610,870) -- cycle ;
	\draw  [fill={rgb, 255:red, 255; green, 255; blue, 255 }  ,fill opacity=1 ] (610,870) -- (680,870) -- (680,910) -- (610,910) -- cycle ;
	\draw  [fill={rgb, 255:red, 255; green, 255; blue, 255 }  ,fill opacity=1 ] (440,590) -- (510,590) -- (510,630) -- (440,630) -- cycle ;
	\draw  [fill={rgb, 255:red, 255; green, 255; blue, 255 }  ,fill opacity=1 ] (610,910) -- (680,910) -- (680,950) -- (610,950) -- cycle ;
	\draw (445,180) node [anchor=north west][inner sep=0.75pt]   [align=left] {$B12$};
	\draw (445,220) node [anchor=north west][inner sep=0.75pt]   [align=left] {$B13$};
	\draw (445,260) node [anchor=north west][inner sep=0.75pt]   [align=left] {$B14$};
	\draw (445,300) node [anchor=north west][inner sep=0.75pt]   [align=left] {$B15$};
	\draw (445,340) node [anchor=north west][inner sep=0.75pt]   [align=left] {$A8$};
	\draw (445,380) node [anchor=north west][inner sep=0.75pt]   [align=left] {$A9$};
	\draw (445,420) node [anchor=north west][inner sep=0.75pt]   [align=left] {$A10$};
	\draw (445,460) node [anchor=north west][inner sep=0.75pt]   [align=left] {$A11$};
	\draw (445,500) node [anchor=north west][inner sep=0.75pt]   [align=left] {$A12$};
	\draw (445,540) node [anchor=north west][inner sep=0.75pt]   [align=left] {$A15$};
	\draw (445,580) node [anchor=north west][inner sep=0.75pt]   [align=left] {$B3$};
	\draw (445,620) node [anchor=north west][inner sep=0.75pt]   [align=left] {$B4$};
	\draw (445,660) node [anchor=north west][inner sep=0.75pt]   [align=left] {$B5$};
	\draw (445,700) node [anchor=north west][inner sep=0.75pt]   [align=left] {$B6$};
	\draw (445,740) node [anchor=north west][inner sep=0.75pt]   [align=left] {$B7$};
	\draw (445,780) node [anchor=north west][inner sep=0.75pt]   [align=left] {$B8$};
	\draw (445,820) node [anchor=north west][inner sep=0.75pt]   [align=left] {$B9$};
	\draw (445,860) node [anchor=north west][inner sep=0.75pt]   [align=left] {$5V$};
	\draw (445,900) node [anchor=north west][inner sep=0.75pt]   [align=left] {$G$};
	\draw (445,940) node [anchor=north west][inner sep=0.75pt]   [align=left] {$3V3$};
	\draw (620,180) node [anchor=north west][inner sep=0.75pt]   [align=left] {$5V$};
	\draw (620,220) node [anchor=north west][inner sep=0.75pt]   [align=left] {$G$};
	\draw (620,260) node [anchor=north west][inner sep=0.75pt]   [align=left] {$3V3$};
	\draw (620,300) node [anchor=north west][inner sep=0.75pt]   [align=left] {$B10$};
	\draw (620,340) node [anchor=north west][inner sep=0.75pt]   [align=left] {$B2$};
	\draw (620,380) node [anchor=north west][inner sep=0.75pt]   [align=left] {$B1$};
	\draw (620,420) node [anchor=north west][inner sep=0.75pt]   [align=left] {$B0$};
	\draw (620,460) node [anchor=north west][inner sep=0.75pt]   [align=left] {$A7$};
	\draw (620,500) node [anchor=north west][inner sep=0.75pt]   [align=left] {$A6$};
	\draw (620,540) node [anchor=north west][inner sep=0.75pt]   [align=left] {$A5$};
	\draw (620,580) node [anchor=north west][inner sep=0.75pt]   [align=left] {$A4$};
	\draw (620,620) node [anchor=north west][inner sep=0.75pt]   [align=left] {$A3$};
	\draw (620,660) node [anchor=north west][inner sep=0.75pt]   [align=left] {$A2$};
	\draw (620,700) node [anchor=north west][inner sep=0.75pt]   [align=left] {$A1$};
	\draw (620,740) node [anchor=north west][inner sep=0.75pt]   [align=left] {$A0$};
	\draw (620,780) node [anchor=north west][inner sep=0.75pt]   [align=left] {$R$};
	\draw (620,820) node [anchor=north west][inner sep=0.75pt]   [align=left] {$C15$};
	\draw (620,860) node [anchor=north west][inner sep=0.75pt]   [align=left] {$C14$};
	\draw (620,900) node [anchor=north west][inner sep=0.75pt]   [align=left] {$C13$};
	\draw (620,940) node [anchor=north west][inner sep=0.75pt]   [align=left] {$VB$};

%			\end{tikzpicture}
%		\end{center}
%		\caption{Pinout del microcontrollore STM32F411CEU6.}
%		\label{fig:pinout_stm32}
%	\end{subfigure}
%	\hfill
%	\begin{subfigure}[t]{0.32\textwidth}
%		\begin{center}
%			\begin{tikzpicture}[x=0.015cm, y=0.015cm, scale=0.5, transform shape]
%				\draw   (120, 90) -- (320,90) -- (320,480) -- (120,480) -- cycle ;

\draw  [fill={rgb, 255:red, 255; green, 255; blue, 255 }  ,fill opacity=1 ]  (90,100)  --  (200,100) -- (200,130) --  (90,130) -- cycle;
\draw  [fill={rgb, 255:red, 255; green, 255; blue, 255 }  ,fill opacity=1 ]  (90,130)  --  (200,130) -- (200,160) --  (90,160) -- cycle;
\draw  [fill={rgb, 255:red, 255; green, 255; blue, 255 }  ,fill opacity=1 ]  (90,160)  --  (200,160) -- (200,190) --  (90,190) -- cycle;
\draw  [fill={rgb, 255:red, 255; green, 255; blue, 255 }  ,fill opacity=1 ]  (90,190)  --  (200,190) -- (200,220) --  (90,220) -- cycle;
\draw  [fill={rgb, 255:red, 255; green, 255; blue, 255 }  ,fill opacity=1 ]  (90,220)  --  (200,220) -- (200,250) --  (90,250) -- cycle;
\draw  [fill={rgb, 255:red, 255; green, 255; blue, 255 }  ,fill opacity=1 ]  (90,250)  --  (200,250) -- (200,280) --  (90,280) -- cycle;
\draw  [fill={rgb, 255:red, 255; green, 255; blue, 255 }  ,fill opacity=1 ]  (90,280)  --  (200,280) -- (200,310) --  (90,310) -- cycle;
\draw  [fill={rgb, 255:red, 255; green, 255; blue, 255 }  ,fill opacity=1 ]  (90,310)  --  (200,310) -- (200,340) --  (90,340) -- cycle;
\draw  [fill={rgb, 255:red, 255; green, 255; blue, 255 }  ,fill opacity=1 ]  (90,340)  --  (200,340) -- (200,370) --  (90,370) -- cycle;
\draw  [fill={rgb, 255:red, 255; green, 255; blue, 255 }  ,fill opacity=1 ]  (90,370)  --  (200,370) -- (200,400) --  (90,400) -- cycle;
\draw  [fill={rgb, 255:red, 255; green, 255; blue, 255 }  ,fill opacity=1 ]  (90,400)  --  (200,400) -- (200,430) --  (90,430) -- cycle;
\draw  [fill={rgb, 255:red, 255; green, 255; blue, 255 }  ,fill opacity=1 ]  (90,430)  --  (200,430) -- (200,460) --  (90,460) -- cycle;

\draw  [fill={rgb, 255:red, 255; green, 255; blue, 255 }  ,fill opacity=1 ] (250,100)  --  (380,100) -- (380,130) --  (250,130) -- cycle;
\draw  [fill={rgb, 255:red, 255; green, 255; blue, 255 }  ,fill opacity=1 ] (250,130)  --  (380,130) -- (380,160) --  (250,160) -- cycle;
\draw  [fill={rgb, 255:red, 255; green, 255; blue, 255 }  ,fill opacity=1 ] (250,160)  --  (380,160) -- (380,190) --  (250,190) -- cycle;
\draw  [fill={rgb, 255:red, 255; green, 255; blue, 255 }  ,fill opacity=1 ] (250,190)  --  (380,190) -- (380,220) --  (250,220) -- cycle;
\draw  [fill={rgb, 255:red, 255; green, 255; blue, 255 }  ,fill opacity=1 ] (250,220)  --  (380,220) -- (380,250) --  (250,250) -- cycle;
\draw  [fill={rgb, 255:red, 255; green, 255; blue, 255 }  ,fill opacity=1 ] (250,250)  --  (380,250) -- (380,280) --  (250,280) -- cycle;
\draw  [fill={rgb, 255:red, 255; green, 255; blue, 255 }  ,fill opacity=1 ] (250,280)  --  (380,280) -- (380,310) --  (250,310) -- cycle;
\draw  [fill={rgb, 255:red, 255; green, 255; blue, 255 }  ,fill opacity=1 ] (250,310)  --  (380,310) -- (380,340) --  (250,340) -- cycle;
\draw  [fill={rgb, 255:red, 255; green, 255; blue, 255 }  ,fill opacity=1 ] (250,340)  --  (380,340) -- (380,370) --  (250,370) -- cycle;


\draw (90,460) node [anchor=north west][inner sep=0.75pt]   [align=left] {$\displaystyle SD\_SCK$};
\draw (90,430) node [anchor=north west][inner sep=0.75pt]   [align=left] {$\displaystyle SD\_DO$};
\draw (90,400) node [anchor=north west][inner sep=0.75pt]   [align=left] {$\displaystyle SD\_DI$};
\draw (90,370) node [anchor=north west][inner sep=0.75pt]   [align=left] {$\displaystyle SD\_SS$};
\draw (90,340) node [anchor=north west][inner sep=0.75pt]   [align=left] {$\displaystyle LCD\_D1$};
\draw (90,310) node [anchor=north west][inner sep=0.75pt]   [align=left] {$\displaystyle LCD\_D0$};
\draw (90,280) node [anchor=north west][inner sep=0.75pt]   [align=left] {$\displaystyle LCD\_D7$};
\draw (90,250) node [anchor=north west][inner sep=0.75pt]   [align=left] {$\displaystyle LCD\_D6$};
\draw (90,220) node [anchor=north west][inner sep=0.75pt]   [align=left] {$\displaystyle LCD\_D4$};
\draw (90,190) node [anchor=north west][inner sep=0.75pt]   [align=left] {$\displaystyle LCD\_D5$};
\draw (90,160) node [anchor=north west][inner sep=0.75pt]   [align=left] {$\displaystyle LCD\_D3$};
\draw (90,130) node [anchor=north west][inner sep=0.75pt]   [align=left] {$\displaystyle LCD\_D2$};

\draw (250,370) node [anchor=north west][inner sep=0.75pt]   [align=left] {$\displaystyle 3.3V$};
\draw (250,340) node [anchor=north west][inner sep=0.75pt]   [align=left] {$\displaystyle 5V$};
\draw (250,310) node [anchor=north west][inner sep=0.75pt]   [align=left] {$\displaystyle GND$};
\draw (250,280) node [anchor=north west][inner sep=0.75pt]   [align=left] {$\displaystyle LCD\_RD$};
\draw (250,250) node [anchor=north west][inner sep=0.75pt]   [align=left] {$\displaystyle LCD\_WR$};
\draw (250,220) node [anchor=north west][inner sep=0.75pt]   [align=left] {$\displaystyle LCD\_RS$};
\draw (250,190) node [anchor=north west][inner sep=0.75pt]   [align=left] {$\displaystyle LCD\_CS$};
\draw (250,160) node [anchor=north west][inner sep=0.75pt]   [align=left] {$\displaystyle LCD\_RST$};
\draw (250,130) node [anchor=north west][inner sep=0.75pt]   [align=left] {$\displaystyle F\_CS$};

%			\end{tikzpicture}
%		\end{center}
%		\caption{Pinout dello schermo \texttt{ILI9341}.}
%		\label{fig:pinout_ili}
%	\end{subfigure}
%	\hfill
%	\begin{subfigure}[t]{0.32\textwidth}
%		\begin{center}
%			\begin{tikzpicture}[x=0.015cm, y=0.015cm, scale=1, transform shape]
%				\draw   (230,40) -- (230,100) -- (190,100) -- (190,130) -- (130,130) -- (130,110) -- (100,110) -- (100,40) -- cycle ;
\draw   (117,86.5) .. controls (117,82.91) and (119.91,80) .. (123.5,80) .. controls (127.09,80) and (130,82.91) .. (130,86.5) .. controls (130,90.09) and (127.09,93) .. (123.5,93) .. controls (119.91,93) and (117,90.09) .. (117,86.5) -- cycle ;
\draw   (130,71.5) .. controls (130,67.91) and (132.91,65) .. (136.5,65) .. controls (140.09,65) and (143,67.91) .. (143,71.5) .. controls (143,75.09) and (140.09,78) .. (136.5,78) .. controls (132.91,78) and (130,75.09) .. (130,71.5) -- cycle ;
\draw   (117,57.5) .. controls (117,53.91) and (119.91,51) .. (123.5,51) .. controls (127.09,51) and (130,53.91) .. (130,57.5) .. controls (130,61.09) and (127.09,64) .. (123.5,64) .. controls (119.91,64) and (117,61.09) .. (117,57.5) -- cycle ;
\draw   (103,71.5) .. controls (103,67.91) and (105.91,65) .. (109.5,65) .. controls (113.09,65) and (116,67.91) .. (116,71.5) .. controls (116,75.09) and (113.09,78) .. (109.5,78) .. controls (105.91,78) and (103,75.09) .. (103,71.5) -- cycle ;
\draw   (192,87.5) .. controls (192,83.91) and (194.91,81) .. (198.5,81) .. controls (202.09,81) and (205,83.91) .. (205,87.5) .. controls (205,91.09) and (202.09,94) .. (198.5,94) .. controls (194.91,94) and (192,91.09) .. (192,87.5) -- cycle ;
\draw   (205,72.5) .. controls (205,68.91) and (207.91,66) .. (211.5,66) .. controls (215.09,66) and (218,68.91) .. (218,72.5) .. controls (218,76.09) and (215.09,79) .. (211.5,79) .. controls (207.91,79) and (205,76.09) .. (205,72.5) -- cycle ;
\draw   (192,58.5) .. controls (192,54.91) and (194.91,52) .. (198.5,52) .. controls (202.09,52) and (205,54.91) .. (205,58.5) .. controls (205,62.09) and (202.09,65) .. (198.5,65) .. controls (194.91,65) and (192,62.09) .. (192,58.5) -- cycle ;
\draw   (178,72.5) .. controls (178,68.91) and (180.91,66) .. (184.5,66) .. controls (188.09,66) and (191,68.91) .. (191,72.5) .. controls (191,76.09) and (188.09,79) .. (184.5,79) .. controls (180.91,79) and (178,76.09) .. (178,72.5) -- cycle ;
\draw   (163,112.5) .. controls (163,108.91) and (165.91,106) .. (169.5,106) .. controls (173.09,106) and (176,108.91) .. (176,112.5) .. controls (176,116.09) and (173.09,119) .. (169.5,119) .. controls (165.91,119) and (163,116.09) .. (163,112.5) -- cycle ;
\draw   (143,112.5) .. controls (143,108.91) and (145.91,106) .. (149.5,106) .. controls (153.09,106) and (156,108.91) .. (156,112.5) .. controls (156,116.09) and (153.09,119) .. (149.5,119) .. controls (145.91,119) and (143,116.09) .. (143,112.5) -- cycle ;

%			\end{tikzpicture}
%		\end{center}
%		\caption{PCB della plancia di gioco.}
%		\label{fig:pinout_pcb}
%	\end{subfigure}
%\end{figure}
Le componenti hardware utilizzate sono tre: il microcontrollore
\texttt{STM32F411CEU6} (\cref{fig:pinout_stm32}), lo schermo \texttt{ILI9341}
(\cref{fig:pinout_ili}) e la PCB per la plancia di gioco
(\cref{fig:pinout_pcb}).

\subsubsection{Microcontrollore}
La componente centrale del progetto è il microcontrollore
\texttt{STM32F411CEU6}, prodotto da ST, montato sulla board \texttt{WeAct Black
	Pill V2.0}. Il microcontrollore monta un core \texttt{arm} Cortex-M4 con clock
massimo $100$MHz. Questo microcontrollore offre $34$ pin GPIO, che verranno
utilizzati per collegare il microcontrollore a schermo e pad; monta $512$KiB di
memoria flash e $128$KiB di SRAM, necessari per l'esecuzione del software
necessario.
\begin{figure}
	\begin{center}
		\begin{tikzpicture}[x=0.015cm, y=0.015cm, scale=0.5, transform shape]
				\draw  [color={rgb, 255:red, 0; green, 0; blue, 0 }  ,draw opacity=1 ][fill={rgb, 255:red, 255; green, 255; blue, 255 }  ,fill opacity=1 ] (480,140) -- (640,140) -- (640,960) -- (480,960) -- cycle ;
	\draw  [fill={rgb, 255:red, 255; green, 255; blue, 255 }  ,fill opacity=1 ] (440,150) -- (510,150) -- (510,190) -- (440,190) -- cycle ;
	\draw  [fill={rgb, 255:red, 255; green, 255; blue, 255 }  ,fill opacity=1 ] (440,190) -- (510,190) -- (510,230) -- (440,230) -- cycle ;
	\draw  [fill={rgb, 255:red, 255; green, 255; blue, 255 }  ,fill opacity=1 ] (440,230) -- (510,230) -- (510,270) -- (440,270) -- cycle ;
	\draw  [fill={rgb, 255:red, 255; green, 255; blue, 255 }  ,fill opacity=1 ] (440,270) -- (510,270) -- (510,310) -- (440,310) -- cycle ;
	\draw  [fill={rgb, 255:red, 255; green, 255; blue, 255 }  ,fill opacity=1 ] (440,310) -- (510,310) -- (510,350) -- (440,350) -- cycle ;
	\draw  [fill={rgb, 255:red, 255; green, 255; blue, 255 }  ,fill opacity=1 ] (440,350) -- (510,350) -- (510,390) -- (440,390) -- cycle ;
	\draw  [fill={rgb, 255:red, 255; green, 255; blue, 255 }  ,fill opacity=1 ] (440,390) -- (510,390) -- (510,430) -- (440,430) -- cycle ;
	\draw  [fill={rgb, 255:red, 255; green, 255; blue, 255 }  ,fill opacity=1 ] (440,430) -- (510,430) -- (510,470) -- (440,470) -- cycle ;
	\draw  [fill={rgb, 255:red, 255; green, 255; blue, 255 }  ,fill opacity=1 ] (440,470) -- (510,470) -- (510,510) -- (440,510) -- cycle ;
	\draw  [fill={rgb, 255:red, 255; green, 255; blue, 255 }  ,fill opacity=1 ] (440,510) -- (510,510) -- (510,550) -- (440,550) -- cycle ;
	\draw  [fill={rgb, 255:red, 255; green, 255; blue, 255 }  ,fill opacity=1 ] (440,550) -- (510,550) -- (510,590) -- (440,590) -- cycle ;
	\draw  [fill={rgb, 255:red, 255; green, 255; blue, 255 }  ,fill opacity=1 ] (440,630) -- (510,630) -- (510,670) -- (440,670) -- cycle ;
	\draw  [fill={rgb, 255:red, 255; green, 255; blue, 255 }  ,fill opacity=1 ] (440,670) -- (510,670) -- (510,710) -- (440,710) -- cycle ;
	\draw  [fill={rgb, 255:red, 255; green, 255; blue, 255 }  ,fill opacity=1 ] (440,710) -- (510,710) -- (510,750) -- (440,750) -- cycle ;
	\draw  [fill={rgb, 255:red, 255; green, 255; blue, 255 }  ,fill opacity=1 ] (440,750) -- (510,750) -- (510,790) -- (440,790) -- cycle ;
	\draw  [fill={rgb, 255:red, 255; green, 255; blue, 255 }  ,fill opacity=1 ] (440,790) -- (510,790) -- (510,830) -- (440,830) -- cycle ;
	\draw  [fill={rgb, 255:red, 255; green, 255; blue, 255 }  ,fill opacity=1 ] (440,830) -- (510,830) -- (510,870) -- (440,870) -- cycle ;
	\draw  [fill={rgb, 255:red, 255; green, 255; blue, 255 }  ,fill opacity=1 ] (440,870) -- (510,870) -- (510,910) -- (440,910) -- cycle ;
	\draw  [fill={rgb, 255:red, 255; green, 255; blue, 255 }  ,fill opacity=1 ] (440,910) -- (510,910) -- (510,950) -- (440,950) -- cycle ;
	\draw  [fill={rgb, 255:red, 255; green, 255; blue, 255 }  ,fill opacity=1 ] (610,150) -- (680,150) -- (680,190) -- (610,190) -- cycle ;
	\draw  [fill={rgb, 255:red, 255; green, 255; blue, 255 }  ,fill opacity=1 ] (610,190) -- (680,190) -- (680,230) -- (610,230) -- cycle ;
	\draw  [fill={rgb, 255:red, 255; green, 255; blue, 255 }  ,fill opacity=1 ] (610,230) -- (680,230) -- (680,270) -- (610,270) -- cycle ;
	\draw  [fill={rgb, 255:red, 255; green, 255; blue, 255 }  ,fill opacity=1 ] (610,270) -- (680,270) -- (680,310) -- (610,310) -- cycle ;
	\draw  [fill={rgb, 255:red, 255; green, 255; blue, 255 }  ,fill opacity=1 ] (610,310) -- (680,310) -- (680,350) -- (610,350) -- cycle ;
	\draw  [fill={rgb, 255:red, 255; green, 255; blue, 255 }  ,fill opacity=1 ] (610,350) -- (680,350) -- (680,390) -- (610,390) -- cycle ;
	\draw  [fill={rgb, 255:red, 255; green, 255; blue, 255 }  ,fill opacity=1 ] (610,390) -- (680,390) -- (680,430) -- (610,430) -- cycle ;
	\draw  [fill={rgb, 255:red, 255; green, 255; blue, 255 }  ,fill opacity=1 ] (610,430) -- (680,430) -- (680,470) -- (610,470) -- cycle ;
	\draw  [fill={rgb, 255:red, 255; green, 255; blue, 255 }  ,fill opacity=1 ] (610,470) -- (680,470) -- (680,510) -- (610,510) -- cycle ;
	\draw  [fill={rgb, 255:red, 255; green, 255; blue, 255 }  ,fill opacity=1 ] (610,510) -- (680,510) -- (680,550) -- (610,550) -- cycle ;
	\draw  [fill={rgb, 255:red, 255; green, 255; blue, 255 }  ,fill opacity=1 ] (610,550) -- (680,550) -- (680,590) -- (610,590) -- cycle ;
	\draw  [fill={rgb, 255:red, 255; green, 255; blue, 255 }  ,fill opacity=1 ] (610,590) -- (680,590) -- (680,630) -- (610,630) -- cycle ;
	\draw  [fill={rgb, 255:red, 255; green, 255; blue, 255 }  ,fill opacity=1 ] (610,630) -- (680,630) -- (680,670) -- (610,670) -- cycle ;
	\draw  [fill={rgb, 255:red, 255; green, 255; blue, 255 }  ,fill opacity=1 ] (610,670) -- (680,670) -- (680,710) -- (610,710) -- cycle ;
	\draw  [fill={rgb, 255:red, 255; green, 255; blue, 255 }  ,fill opacity=1 ] (610,710) -- (680,710) -- (680,750) -- (610,750) -- cycle ;
	\draw  [fill={rgb, 255:red, 255; green, 255; blue, 255 }  ,fill opacity=1 ] (610,750) -- (680,750) -- (680,790) -- (610,790) -- cycle ;
	\draw  [fill={rgb, 255:red, 255; green, 255; blue, 255 }  ,fill opacity=1 ] (610,790) -- (680,790) -- (680,830) -- (610,830) -- cycle ;
	\draw  [fill={rgb, 255:red, 255; green, 255; blue, 255 }  ,fill opacity=1 ] (610,830) -- (680,830) -- (680,870) -- (610,870) -- cycle ;
	\draw  [fill={rgb, 255:red, 255; green, 255; blue, 255 }  ,fill opacity=1 ] (610,870) -- (680,870) -- (680,910) -- (610,910) -- cycle ;
	\draw  [fill={rgb, 255:red, 255; green, 255; blue, 255 }  ,fill opacity=1 ] (440,590) -- (510,590) -- (510,630) -- (440,630) -- cycle ;
	\draw  [fill={rgb, 255:red, 255; green, 255; blue, 255 }  ,fill opacity=1 ] (610,910) -- (680,910) -- (680,950) -- (610,950) -- cycle ;
	\draw (445,180) node [anchor=north west][inner sep=0.75pt]   [align=left] {$B12$};
	\draw (445,220) node [anchor=north west][inner sep=0.75pt]   [align=left] {$B13$};
	\draw (445,260) node [anchor=north west][inner sep=0.75pt]   [align=left] {$B14$};
	\draw (445,300) node [anchor=north west][inner sep=0.75pt]   [align=left] {$B15$};
	\draw (445,340) node [anchor=north west][inner sep=0.75pt]   [align=left] {$A8$};
	\draw (445,380) node [anchor=north west][inner sep=0.75pt]   [align=left] {$A9$};
	\draw (445,420) node [anchor=north west][inner sep=0.75pt]   [align=left] {$A10$};
	\draw (445,460) node [anchor=north west][inner sep=0.75pt]   [align=left] {$A11$};
	\draw (445,500) node [anchor=north west][inner sep=0.75pt]   [align=left] {$A12$};
	\draw (445,540) node [anchor=north west][inner sep=0.75pt]   [align=left] {$A15$};
	\draw (445,580) node [anchor=north west][inner sep=0.75pt]   [align=left] {$B3$};
	\draw (445,620) node [anchor=north west][inner sep=0.75pt]   [align=left] {$B4$};
	\draw (445,660) node [anchor=north west][inner sep=0.75pt]   [align=left] {$B5$};
	\draw (445,700) node [anchor=north west][inner sep=0.75pt]   [align=left] {$B6$};
	\draw (445,740) node [anchor=north west][inner sep=0.75pt]   [align=left] {$B7$};
	\draw (445,780) node [anchor=north west][inner sep=0.75pt]   [align=left] {$B8$};
	\draw (445,820) node [anchor=north west][inner sep=0.75pt]   [align=left] {$B9$};
	\draw (445,860) node [anchor=north west][inner sep=0.75pt]   [align=left] {$5V$};
	\draw (445,900) node [anchor=north west][inner sep=0.75pt]   [align=left] {$G$};
	\draw (445,940) node [anchor=north west][inner sep=0.75pt]   [align=left] {$3V3$};
	\draw (620,180) node [anchor=north west][inner sep=0.75pt]   [align=left] {$5V$};
	\draw (620,220) node [anchor=north west][inner sep=0.75pt]   [align=left] {$G$};
	\draw (620,260) node [anchor=north west][inner sep=0.75pt]   [align=left] {$3V3$};
	\draw (620,300) node [anchor=north west][inner sep=0.75pt]   [align=left] {$B10$};
	\draw (620,340) node [anchor=north west][inner sep=0.75pt]   [align=left] {$B2$};
	\draw (620,380) node [anchor=north west][inner sep=0.75pt]   [align=left] {$B1$};
	\draw (620,420) node [anchor=north west][inner sep=0.75pt]   [align=left] {$B0$};
	\draw (620,460) node [anchor=north west][inner sep=0.75pt]   [align=left] {$A7$};
	\draw (620,500) node [anchor=north west][inner sep=0.75pt]   [align=left] {$A6$};
	\draw (620,540) node [anchor=north west][inner sep=0.75pt]   [align=left] {$A5$};
	\draw (620,580) node [anchor=north west][inner sep=0.75pt]   [align=left] {$A4$};
	\draw (620,620) node [anchor=north west][inner sep=0.75pt]   [align=left] {$A3$};
	\draw (620,660) node [anchor=north west][inner sep=0.75pt]   [align=left] {$A2$};
	\draw (620,700) node [anchor=north west][inner sep=0.75pt]   [align=left] {$A1$};
	\draw (620,740) node [anchor=north west][inner sep=0.75pt]   [align=left] {$A0$};
	\draw (620,780) node [anchor=north west][inner sep=0.75pt]   [align=left] {$R$};
	\draw (620,820) node [anchor=north west][inner sep=0.75pt]   [align=left] {$C15$};
	\draw (620,860) node [anchor=north west][inner sep=0.75pt]   [align=left] {$C14$};
	\draw (620,900) node [anchor=north west][inner sep=0.75pt]   [align=left] {$C13$};
	\draw (620,940) node [anchor=north west][inner sep=0.75pt]   [align=left] {$VB$};

		\end{tikzpicture}
	\end{center}
	\caption{Pinout del microcontrollore STM32F411CEU6.}
	\label{fig:pinout_stm32}

\end{figure}

\subsubsection{Schermo}
Lo schermo che abbiamo utilizzato, chiamato \texttt{ILI9341}, è uno schermo LCD
da $2.8''$, con una risoluzione di $320x240$ e controllo individuale dei pixel.
Una particolarità di questo schermo è che offre anche la possibilità di inserire
una scheda microSD al suo interno, che utilizziamo come memoria per i giochi da
utilizzare. 
\begin{figure}[h]
	\begin{center}
		\begin{tikzpicture}[x=0.015cm, y=0.015cm, scale=0.5, transform shape]
			\draw   (120, 90) -- (320,90) -- (320,480) -- (120,480) -- cycle ;

\draw  [fill={rgb, 255:red, 255; green, 255; blue, 255 }  ,fill opacity=1 ]  (90,100)  --  (200,100) -- (200,130) --  (90,130) -- cycle;
\draw  [fill={rgb, 255:red, 255; green, 255; blue, 255 }  ,fill opacity=1 ]  (90,130)  --  (200,130) -- (200,160) --  (90,160) -- cycle;
\draw  [fill={rgb, 255:red, 255; green, 255; blue, 255 }  ,fill opacity=1 ]  (90,160)  --  (200,160) -- (200,190) --  (90,190) -- cycle;
\draw  [fill={rgb, 255:red, 255; green, 255; blue, 255 }  ,fill opacity=1 ]  (90,190)  --  (200,190) -- (200,220) --  (90,220) -- cycle;
\draw  [fill={rgb, 255:red, 255; green, 255; blue, 255 }  ,fill opacity=1 ]  (90,220)  --  (200,220) -- (200,250) --  (90,250) -- cycle;
\draw  [fill={rgb, 255:red, 255; green, 255; blue, 255 }  ,fill opacity=1 ]  (90,250)  --  (200,250) -- (200,280) --  (90,280) -- cycle;
\draw  [fill={rgb, 255:red, 255; green, 255; blue, 255 }  ,fill opacity=1 ]  (90,280)  --  (200,280) -- (200,310) --  (90,310) -- cycle;
\draw  [fill={rgb, 255:red, 255; green, 255; blue, 255 }  ,fill opacity=1 ]  (90,310)  --  (200,310) -- (200,340) --  (90,340) -- cycle;
\draw  [fill={rgb, 255:red, 255; green, 255; blue, 255 }  ,fill opacity=1 ]  (90,340)  --  (200,340) -- (200,370) --  (90,370) -- cycle;
\draw  [fill={rgb, 255:red, 255; green, 255; blue, 255 }  ,fill opacity=1 ]  (90,370)  --  (200,370) -- (200,400) --  (90,400) -- cycle;
\draw  [fill={rgb, 255:red, 255; green, 255; blue, 255 }  ,fill opacity=1 ]  (90,400)  --  (200,400) -- (200,430) --  (90,430) -- cycle;
\draw  [fill={rgb, 255:red, 255; green, 255; blue, 255 }  ,fill opacity=1 ]  (90,430)  --  (200,430) -- (200,460) --  (90,460) -- cycle;

\draw  [fill={rgb, 255:red, 255; green, 255; blue, 255 }  ,fill opacity=1 ] (250,100)  --  (380,100) -- (380,130) --  (250,130) -- cycle;
\draw  [fill={rgb, 255:red, 255; green, 255; blue, 255 }  ,fill opacity=1 ] (250,130)  --  (380,130) -- (380,160) --  (250,160) -- cycle;
\draw  [fill={rgb, 255:red, 255; green, 255; blue, 255 }  ,fill opacity=1 ] (250,160)  --  (380,160) -- (380,190) --  (250,190) -- cycle;
\draw  [fill={rgb, 255:red, 255; green, 255; blue, 255 }  ,fill opacity=1 ] (250,190)  --  (380,190) -- (380,220) --  (250,220) -- cycle;
\draw  [fill={rgb, 255:red, 255; green, 255; blue, 255 }  ,fill opacity=1 ] (250,220)  --  (380,220) -- (380,250) --  (250,250) -- cycle;
\draw  [fill={rgb, 255:red, 255; green, 255; blue, 255 }  ,fill opacity=1 ] (250,250)  --  (380,250) -- (380,280) --  (250,280) -- cycle;
\draw  [fill={rgb, 255:red, 255; green, 255; blue, 255 }  ,fill opacity=1 ] (250,280)  --  (380,280) -- (380,310) --  (250,310) -- cycle;
\draw  [fill={rgb, 255:red, 255; green, 255; blue, 255 }  ,fill opacity=1 ] (250,310)  --  (380,310) -- (380,340) --  (250,340) -- cycle;
\draw  [fill={rgb, 255:red, 255; green, 255; blue, 255 }  ,fill opacity=1 ] (250,340)  --  (380,340) -- (380,370) --  (250,370) -- cycle;


\draw (90,460) node [anchor=north west][inner sep=0.75pt]   [align=left] {$\displaystyle SD\_SCK$};
\draw (90,430) node [anchor=north west][inner sep=0.75pt]   [align=left] {$\displaystyle SD\_DO$};
\draw (90,400) node [anchor=north west][inner sep=0.75pt]   [align=left] {$\displaystyle SD\_DI$};
\draw (90,370) node [anchor=north west][inner sep=0.75pt]   [align=left] {$\displaystyle SD\_SS$};
\draw (90,340) node [anchor=north west][inner sep=0.75pt]   [align=left] {$\displaystyle LCD\_D1$};
\draw (90,310) node [anchor=north west][inner sep=0.75pt]   [align=left] {$\displaystyle LCD\_D0$};
\draw (90,280) node [anchor=north west][inner sep=0.75pt]   [align=left] {$\displaystyle LCD\_D7$};
\draw (90,250) node [anchor=north west][inner sep=0.75pt]   [align=left] {$\displaystyle LCD\_D6$};
\draw (90,220) node [anchor=north west][inner sep=0.75pt]   [align=left] {$\displaystyle LCD\_D4$};
\draw (90,190) node [anchor=north west][inner sep=0.75pt]   [align=left] {$\displaystyle LCD\_D5$};
\draw (90,160) node [anchor=north west][inner sep=0.75pt]   [align=left] {$\displaystyle LCD\_D3$};
\draw (90,130) node [anchor=north west][inner sep=0.75pt]   [align=left] {$\displaystyle LCD\_D2$};

\draw (250,370) node [anchor=north west][inner sep=0.75pt]   [align=left] {$\displaystyle 3.3V$};
\draw (250,340) node [anchor=north west][inner sep=0.75pt]   [align=left] {$\displaystyle 5V$};
\draw (250,310) node [anchor=north west][inner sep=0.75pt]   [align=left] {$\displaystyle GND$};
\draw (250,280) node [anchor=north west][inner sep=0.75pt]   [align=left] {$\displaystyle LCD\_RD$};
\draw (250,250) node [anchor=north west][inner sep=0.75pt]   [align=left] {$\displaystyle LCD\_WR$};
\draw (250,220) node [anchor=north west][inner sep=0.75pt]   [align=left] {$\displaystyle LCD\_RS$};
\draw (250,190) node [anchor=north west][inner sep=0.75pt]   [align=left] {$\displaystyle LCD\_CS$};
\draw (250,160) node [anchor=north west][inner sep=0.75pt]   [align=left] {$\displaystyle LCD\_RST$};
\draw (250,130) node [anchor=north west][inner sep=0.75pt]   [align=left] {$\displaystyle F\_CS$};

		\end{tikzpicture}
	\end{center}
	\caption{Pinout dello schermo \texttt{ILI9341}.}
	\label{fig:pinout_ili}

\end{figure}

\subsubsection{Gamepad}
L'ultimo componente hardware necessario è la plancia di gioco: per questo
utilizziamo una board costruita appositamente per il form factor del GameBoy
originale, con però $12$ tasti, di cui noi ne utilizzeremo solo $8$: frecce
direzionali, \texttt{a}, \texttt{b}, \texttt{select} e \texttt{start}.
\begin{figure}[h]
	\begin{center}
		\begin{tikzpicture}[x=0.75pt, y=0.75pt, scale=1]
			\draw   (230,40) -- (230,100) -- (190,100) -- (190,130) -- (130,130) -- (130,110) -- (100,110) -- (100,40) -- cycle ;
\draw   (117,86.5) .. controls (117,82.91) and (119.91,80) .. (123.5,80) .. controls (127.09,80) and (130,82.91) .. (130,86.5) .. controls (130,90.09) and (127.09,93) .. (123.5,93) .. controls (119.91,93) and (117,90.09) .. (117,86.5) -- cycle ;
\draw   (130,71.5) .. controls (130,67.91) and (132.91,65) .. (136.5,65) .. controls (140.09,65) and (143,67.91) .. (143,71.5) .. controls (143,75.09) and (140.09,78) .. (136.5,78) .. controls (132.91,78) and (130,75.09) .. (130,71.5) -- cycle ;
\draw   (117,57.5) .. controls (117,53.91) and (119.91,51) .. (123.5,51) .. controls (127.09,51) and (130,53.91) .. (130,57.5) .. controls (130,61.09) and (127.09,64) .. (123.5,64) .. controls (119.91,64) and (117,61.09) .. (117,57.5) -- cycle ;
\draw   (103,71.5) .. controls (103,67.91) and (105.91,65) .. (109.5,65) .. controls (113.09,65) and (116,67.91) .. (116,71.5) .. controls (116,75.09) and (113.09,78) .. (109.5,78) .. controls (105.91,78) and (103,75.09) .. (103,71.5) -- cycle ;
\draw   (192,87.5) .. controls (192,83.91) and (194.91,81) .. (198.5,81) .. controls (202.09,81) and (205,83.91) .. (205,87.5) .. controls (205,91.09) and (202.09,94) .. (198.5,94) .. controls (194.91,94) and (192,91.09) .. (192,87.5) -- cycle ;
\draw   (205,72.5) .. controls (205,68.91) and (207.91,66) .. (211.5,66) .. controls (215.09,66) and (218,68.91) .. (218,72.5) .. controls (218,76.09) and (215.09,79) .. (211.5,79) .. controls (207.91,79) and (205,76.09) .. (205,72.5) -- cycle ;
\draw   (192,58.5) .. controls (192,54.91) and (194.91,52) .. (198.5,52) .. controls (202.09,52) and (205,54.91) .. (205,58.5) .. controls (205,62.09) and (202.09,65) .. (198.5,65) .. controls (194.91,65) and (192,62.09) .. (192,58.5) -- cycle ;
\draw   (178,72.5) .. controls (178,68.91) and (180.91,66) .. (184.5,66) .. controls (188.09,66) and (191,68.91) .. (191,72.5) .. controls (191,76.09) and (188.09,79) .. (184.5,79) .. controls (180.91,79) and (178,76.09) .. (178,72.5) -- cycle ;
\draw   (163,112.5) .. controls (163,108.91) and (165.91,106) .. (169.5,106) .. controls (173.09,106) and (176,108.91) .. (176,112.5) .. controls (176,116.09) and (173.09,119) .. (169.5,119) .. controls (165.91,119) and (163,116.09) .. (163,112.5) -- cycle ;
\draw   (143,112.5) .. controls (143,108.91) and (145.91,106) .. (149.5,106) .. controls (153.09,106) and (156,108.91) .. (156,112.5) .. controls (156,116.09) and (153.09,119) .. (149.5,119) .. controls (145.91,119) and (143,116.09) .. (143,112.5) -- cycle ;

		\end{tikzpicture}
	\end{center}
	\caption{PCB della plancia di gioco.}
	\label{fig:pinout_pcb}
\end{figure}



\subsection{Software}
\subsubsection{Emulatore}
L'emulatore che abbiamo scelto di portare è un emulatore scritto in C (header
only) per desktop, chiamato \texttt{peanutGB}. 
\end{document}
