\documentclass[12pt]{article}

\usepackage{hyperref}
\usepackage[italian]{babel}
\usepackage[a4paper, total={6in, 8in}]{geometry}

\title{STM32 GameBoy}
\date{\today}
\author{Marco Cutecchia, Edoardo Marangoni}


\begin{document}
\maketitle
\begin{center}\rule{0.5\linewidth}{0.5pt}\end{center}

\section{Introduzione}

Il progetto consiste nel creare un
\href{https://it.wikipedia.org/wiki/Game_Boy}{GameBoy} utilizzando una
scheda STM32. Il risultato sarà composto dalle parti necessarie per
riprodurre l'esperienza originale del GameBoy, partendo ovviamente
dall'emulatore sino al utilizzo di un case fedele al form factor
originale. L'unica deviazione dalla riproduzione esatta sarà quella
delle \emph{cartridge}: infatti, mentre in origine i giochi erano
caricati sul GameBoy tramite un supporto rimovibile contenente la ROM,
nella nostra realizzazione vi saranno a disposizione più ROM di giochi
inseriti in una singola scheda SD, dalla quale la board potrà leggere i
dati di gioco.

\subsection{Costruzione}

\subsubsection{Hardware}

Il componente principale del progetto è la scheda STM32
\texttt{stm32f411ceu6}, nota anche come \emph{WeAct BlackPill}.
L'obiettivo finale è quello di mostrare all'utente un gioco, il quale
verrà visualizzato su uno schermo: abbiamo scelto lo schermo
\texttt{ili9341\ tft}, da 2.4'\,', il quale è anche dotato di una porta
per schede microSD. L'utente, quindi, avrà modo di interagire con
l'emulatore utilizzando la plancia di comando che sarà fedele al GameBoy
originale, realizzata con un circuito stampato che ricalca la posizione
dei tasti nel form factor desiderato. Il tutto andrà inserito in un
\emph{guscio} nella forma originale.

\subsubsection{Software}

Grazie alla disponibilità online di svariate implementazioni di
emulatori di GameBoy, adatteremo una versione open source per desktop
alle necessità e ai limiti di un microcontrollore. In secondo luogo,
implementeremo un driver per interfacciarci con lo schermo che abbiamo
scelto, sia per mostrare il video che per leggere dalla microSD. Infine,
sarà necessario gestire i segnali inviati dalla plancia di gioco,
passandoli all'emulatore. Non ultimo, sarà importante gestire il caching
delle ROM dei giochi per rendere l'emulatore più veloce.

\subsubsection{Costi}


\begin{table}[h]
\begin{tabular}{|llll|l|}
\hline
\multicolumn{1}{|l|}{Nome}             & \multicolumn{1}{l|}{Modello}                           & \multicolumn{1}{l|}{Costo unitario} & Unità & Costo \\ \hline
\multicolumn{1}{|l|}{Guscio esterno}   & \multicolumn{1}{l|}{GameBoy DMG-01 housing shell}      & \multicolumn{1}{l|}{xxx}            & 1     & xxx   \\ \hline
\multicolumn{1}{|l|}{Plancia di gioco} & \multicolumn{1}{l|}{GameBoy DMG-01 PCB}                & \multicolumn{1}{l|}{xxx}            & 1     & xxx   \\ \hline
\multicolumn{1}{|l|}{Bottoni}          & \multicolumn{1}{l|}{GameBoy DMG-01 rubber button pads} & \multicolumn{1}{l|}{xxx}            & 1     & xxx   \\ \hline
\multicolumn{1}{|l|}{Schermo}          & \multicolumn{1}{l|}{ili9341 tft 2.4''}                 & \multicolumn{1}{l|}{xxx}            & 1     & xxx   \\ \hline
\multicolumn{1}{|l|}{Microcontrollore} & \multicolumn{1}{l|}{stm32f411ceu6}                     & \multicolumn{1}{l|}{xxx}            & 1     & xxx   \\ \hline
\multicolumn{4}{|l|}{Totale}                                                                                                                  &       \\ \hline
\end{tabular}
\end{table}
\end{document}
