\documentclass[12pt]{article}

\usepackage{hyperref}
\usepackage[italian]{babel}
\usepackage[a4paper]{geometry}

\setlength{\parskip}{1.2ex}
\setlength{\parindent}{0em}
\clubpenalty = 100
\widowpenalty = 100

\title{STM32 GameBoy}
\date{\today}
\author{Marco Cutecchia, Edoardo Marangoni \\
\footnotesize \texttt{\{marco.cutecchia, edoardo.marangoni\}@studenti.unimi.it} \\ }

\begin{document}
\maketitle

\section{Introduzione}
Il GameBoy è una console portatile rilasciata da Nintendo all'inizio degli anni
90 che diede inizio al grandissimo successo dei videogiochi tascabili.
La console vendette più di 118 milioni di unità nel mondo e divenne un fenomeno
culturale ricordato ancora oggi.

Lo scopo del progetto è quello di costruire un'imitazione del GameBoy in grado
di giocare tutti i videogiochi originali, ma con una serie di miglioramenti
grazie ai componenti hardware più moderni.
Tra le modifiche al dispositivo originale che siamo intenzionati ad implementare
troviamo il caricamento dei giochi tramite microSD, invece che con cartucce,
e l'utilizzo di uno schermo a colori e retroilluminato, al contrario dell'originale
schermo a scala di grigi che diventava impossibile da vedere sotto scarsa luce.

Benché questo progetto sia realizzato ad uno scopo puramente creativo, può
servire ad appassionati di retrogaming come prototipo per una propria
realizzazione.
In rete è possibile trovare diverse imitazioni del GameBoy, ma molte di
queste o non permettono di giocare veramente ai giochi originali oppure
integrano al loro interno interi computer esageratamente potenti per il
compito.

\section{Componenti Hardware}
Il componente principale del progetto è una scheda basata sul chip
\texttt{STM32 F411CEU6}.
Questa scheda è stata scelta per due motivi: innanzitutto le sue dimensioni sono
tali da permetterne l'inserimento nel case originale.
In secondo luogo, la scheda utilizzata durante il corso
(\texttt{STM32L053R8} \textit{Nucleo-64}) non è sufficientemente potente per
poter emulare un GameBoy.
Infatti, tramite i tool offerti da STM32CubeIDE, abbiamo trovato che sono
necessari almeno $70$ KB di memoria RAM per poter eseguire solamente una versione
ridotta dell'emulatore; un valore ben più alto dei $8$ KB di SRAM
dell'\texttt{STM32L053R8}.

Chiaramente sarà necessario un display dove mostrare il gioco in esecuzione.
I requisiti che ci siamo posti sono quelli di uno schermo a colori,
retroilluminato e di dimensioni ridotte in modo da poter essere inserito
al posto dello schermo originale del GameBoy.
Sotto queste condizioni abbiamo scelto un display TFT da $2.4$ pollici con
risoluzione 320x240 basato sul controller \texttt{ILI9341}.
Convenientemente, questo tipo di display integra anche una porta microSD
che fa al nostro scopo.

Infine avremo bisogno dei tasti, di un case di GameBoy dove alloggiare i
componenti e di un interruttore per controllare l'alimentazione
(via batterie AA) al sistema.
Fortunatamente il GameBoy gode ancora di una attiva comunità di modding,
di conseguenza pezzi di ricambio vengono ancora prodotti e sono disponibili
a prezzi molto bassi.

\section{Software da Implementare}
Grazie alla disponibilità online di svariate implementazioni di emulatori di
GameBoy, adatteremo una versione open source per desktop alle necessità e ai
limiti di un microcontrollore. L'emulatore che intendiamo portare è $PeanutGB$,
scelto perchè ha dipendenze minime e ben separate dal core dell'emulatore.

In secondo luogo, implementeremo un driver per interfacciarci con lo schermo
che abbiamo scelto: seppur il display scelto sia relativamente popolare
le librerie per controllarlo compatibili con STM32 sono poche, inoltre spesso
utilizzano il protocollo SPI per il trasferimento dati invece della più veloce
interfaccia seriale ad 8 bit.

Il caricamento dei videogiochi da microSD verrà fatto utilizzando l'interfaccia
SPI. Intendiamo avvalerci dalle librerie ufficiali ST per semplificare
l'implementazione del trasferimento dati ma anche per la gestione del filesystem
FAT32. Nonostante questo prevediamo che sarà importante implementare un
meccanismo di caching parziale dei dati letti dalla microSD per rendere
l'emulazione più veloce, dato che non tutti i giochi per GameBoy possono essere
caricati interamente in memoria RAM (alcuni di essi arrivano fino a 8MB).

Sarà necessario implementare anche una GUI per la selezione del gioco da far
partire. Vista la relativa semplicità dell'interfaccia che intendiamo realizzare
e le scarse risorse computazionali a nostra disposizione, ci sembra inopportuno
includere un framework come TouchGFX.
Di conseguenza intendiamo implementare "a mano" un semplice renderer di font ed
alcune primitive per il disegno su framebuffer.

Infine sarà necessario gestire i segnali inviati dalla plancia di gioco,
passandoli all'emulatore tramite GPIO.
Non prevediamo particolari difficoltà con quest ultimo compito. 

\section{Analisi di Costi e Tempi}

\begin{center}
\begin{table}[h]
    \centering
    \begin{tabular}{|llll|l|}
        \hline
        \multicolumn{1}{|l|}{\textbf{Nome}}    & \multicolumn{1}{l|}{\textbf{Modello}}  & \multicolumn{1}{l|}{\textbf{Costo unitario}} & \textbf{Unità} & \textbf{Costo} \\ \hline
        \multicolumn{1}{|l|}{Guscio esterno}   & \multicolumn{1}{l|}{GB DMG-01 Shell}   & \multicolumn{1}{l|}{9.90}                     & 1              & 9.90          \\ \hline
        \multicolumn{1}{|l|}{Plancia di gioco} & \multicolumn{1}{l|}{GB DMG-01 PCB}     & \multicolumn{1}{l|}{1.18}                     & 1              & 1.18          \\ \hline
        \multicolumn{1}{|l|}{Bottoni}          & \multicolumn{1}{l|}{GB DMG-01 Buttons} & \multicolumn{1}{l|}{2.95}                     & 1              & 2.95          \\ \hline
        \multicolumn{1}{|l|}{Schermo}          & \multicolumn{1}{l|}{ILI9341 2.4"}      & \multicolumn{1}{l|}{6.44}                     & 1              & 6.44          \\ \hline
        \multicolumn{1}{|l|}{Microcontrollore} & \multicolumn{1}{l|}{STM32 F411CEU6}    & \multicolumn{1}{l|}{7.03}                     & 1              & 7.03          \\ \hline
        \multicolumn{1}{|l|}{Interruttore}     & \multicolumn{1}{l|}{SS12D00 4mm}       & \multicolumn{1}{l|}{0.30}                     & 1              & 0.30          \\ \hline
        \multicolumn{4}{|r|}{\textbf{Totale}}                                                                                                            & 27,8€         \\ \hline
    \end{tabular}
    \caption{
        Materiali previsti per la costruzione del progetto. I costi indicati
        provengono da negozi online come Amazon, eBay e Aliexpress
    }
\end{table}
\end{center}

\begin{center}
\begin{table}[h]
    \centering
    \begin{tabular}{|l|l|}
        \hline
        \textbf{Nome}                         & \textbf{Tempo}  \\ \hline
        Porting Emulatore                     & 4 giorni        \\ \hline
        Driver Display                        & 4 giorni        \\ \hline
        Integrazione PCB-Emulatore            & 1 giorno        \\ \hline
        Caricamento Videogiochi da SD         & 2 giorni        \\ \hline
        Interfaccia Selezione Videogioco      & 4 giorni        \\ \hline
        Ottimizzazione Software               & 2 settimane     \\ \hline
        Sistema di Alimentazione              & 2 giorni        \\ \hline
        \multicolumn{1}{|r|}{\textbf{Totale}} & 1 mese circa    \\ \hline
    \end{tabular}
    \caption{
        Tempi di realizzazione previsti per la realizzazione dei componenti software
    }
\end{table}

\end{center}
\end{document}
