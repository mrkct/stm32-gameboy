\documentclass[12pt]{article}

\usepackage{hyperref}
\usepackage[italian]{babel}
\usepackage[a4paper]{geometry}

\title{STM32 GameBoy}
\date{\today}
\author{Marco Cutecchia, Edoardo Marangoni}

\begin{document}
\maketitle
\begin{center}\rule{0.5\linewidth}{0.5pt}\end{center}

\section{Introduzione}
Il GameBoy è una console portatile rilasciata da Nintendo all'inizio degli anni
'90 che diede inizio al grandissimo successo dei videogiochi tascabili. Il
GameBoy vendette più di 118 milioni di unità nel mondo e divenne un fenomeno
culturale ricordato ancora oggi.

In origine, il GameBoy caricava i giochi tramite cartucce ingombranti: un
giocatore, pertanto, doveva avere con sé una cartuccia per ogni singolo gioco.
Inoltre, date le possibilità in termini di costo e dimensioni del tempo, il
GameBoy era dotato di uno schermo a cristalli liquidi monocolore e senza
retroilluminazione. 

Lo scopo del progetto è quello di costruire un'imitazione del GameBoy in grado
di giocare tutti i videogiochi originali, ma con una serie di miglioramenti
dovuti ai componenti hardware più moderni: utilizzeremo una microSD per
contenere le ROM dei giochi ed useremo uno schermo a colori con risoluzione più
alta e retroilluminato.

\subsection{Costruzione}
\subsubsection{Hardware}
Il componente principale del progetto è la scheda \emph{WeAct BlackPill}, basata
sul SOC STM32 \texttt{stm32f411ceu6}. Questa scheda è stata scelta per due
motivi: innanzitutto le sue dimensioni sono tali da permetterne l'inserimento
nel case originale. In secondo luogo, la scheda utilizzata durante il corso
(\texttt{stm32l053r8} \textit{Nucleo}), essendo dotata di $8$ KB di SRAM, porta
a dei problemi durante l'esecuzione di alcune funzioni che, secondo l'analisi
statica di STM32CubeIde, utilizzano fino a $70$ KB di memoria statica: la scheda scelta,
invece, dispone di $120$ KB di SRAM, che sono sufficienti.

L'obiettivo finale è quello di mostrare all'utente un gioco, il quale verrà
visualizzato su uno schermo: abbiamo scelto lo schermo \texttt{ili9341 tft}, da
$2.4$ pollici, il quale è anche dotato di una porta per schede microSD e offre
un'interfaccia parallela a $8$ bit. L'utente, quindi, avrà modo di interagire
con l'emulatore utilizzando la plancia di comando che sarà fedele al GameBoy
originale, realizzata con un circuito stampato che ricalca la posizione dei
tasti nel form factor desiderato. Come anticipato, l tutto andrà inserito in un
\emph{guscio} nella forma originale.

\subsubsection{Software}
Grazie alla disponibilità online di svariate implementazioni di emulatori di
GameBoy, adatteremo una versione open source per desktop alle necessità e ai
limiti di un microcontrollore. In secondo luogo, implementeremo un driver per
interfacciarci con lo schermo che abbiamo scelto, sia per mostrare il video che
per leggere dalla microSD, utilizzando l'interfaccia SPI. Infine, sarà
necessario gestire i segnali inviati dalla plancia di gioco, passandoli
all'emulatore tramite GPIO. Non ultimo, sarà importante gestire il caching delle
ROM dei giochi per rendere l'emulatore più veloce, con l'ausilio di un profiler
per garantire un'esperienza fluida al giocatore.

\subsubsection{Utilità}
Benché questo progetto sia realizzato ad uno scopo puramente creativo, può
servire ad appassionati di retrogaming come prototipo per una propria
realizzazione, in quanto molte delle imitazioni che si trovano online, benché
mantengano il \textit{form factor} del GameBoy originale, non permettono di
giocare ai giochi originali. 

\subsubsection{Costi}
\begin{center}
\begin{table}[h]
\begin{tabular}{|llll|l|}
\hline
\multicolumn{1}{|l|}{\textbf{Nome}}    & \multicolumn{1}{l|}{\textbf{Modello}}             & \multicolumn{1}{l|}{\textbf{Costo unitario}} & \textbf{Unità} & \textbf{Costo} \\ \hline
\multicolumn{1}{|l|}{Guscio esterno}   & \multicolumn{1}{l|}{GB DMG-01 shell}      & \multicolumn{1}{l|}{xxx}                     & 1              & xxx            \\ \hline
\multicolumn{1}{|l|}{Plancia di gioco} & \multicolumn{1}{l|}{GB DMG-01 PCB}                & \multicolumn{1}{l|}{xxx}                     & 1              & xxx            \\ \hline
\multicolumn{1}{|l|}{Bottoni}          & \multicolumn{1}{l|}{GB DMG-01 buttons} & \multicolumn{1}{l|}{xxx}                     & 1              & xxx            \\ \hline
\multicolumn{1}{|l|}{Schermo}          & \multicolumn{1}{l|}{ili9341 tft 2.4''}            & \multicolumn{1}{l|}{xxx}                     & 1              & xxx            \\ \hline
\multicolumn{1}{|l|}{Microcontrollore} & \multicolumn{1}{l|}{stm32f411ceu6}                & \multicolumn{1}{l|}{xxx}                     & 1              & xxx            \\ \hline
\multicolumn{4}{|r|}{\textbf{Totale}}                                                                                                                               &                \\ \hline
\end{tabular}
\end{table}
\end{center}

\subsubsection{Tempo}
\begin{center}
\begin{table}[h]
\begin{tabular}{|l|l|}
\hline
\textbf{Nome}                         & \textbf{Tempo} \\ \hline
porting emulatore                     & xxx            \\ \hline
driver display                        & xxx            \\ \hline
integrazione PCB-emulatore            & xxx            \\ \hline
caricamento videogiochi da SD         & xxx            \\ \hline
interfaccia selezione videogioco      & xxx            \\ \hline
ottimizzazione software               & xxx            \\ \hline
sistema di alimentazione              & xxx            \\ \hline
\multicolumn{1}{|r|}{\textbf{Totale}} & xxx            \\ \hline
\end{tabular}
\end{table}

\end{center}
\end{document}
